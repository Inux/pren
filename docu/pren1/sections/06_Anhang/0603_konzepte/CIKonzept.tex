\documentclass[../../../main.tex]{subfiles}
\begin{document}
\subsection{Continuous Integration Konzept}
Um die Software zu entwickeln und die Dokumentation zu schreiben verwenden wir ein Git repository
welches auf github.com gehostet wird. Die Dokumentation wird mithilfe von LaTeX geschrieben.
Beide Entscheidungen wurden früh im Projektverlauf getroffen basierend auf einer Teamdiskussion.
Um zu gewährleisten das die Software zu jeder Zeit funktioniert haben wir uns dazu entschiedenen einen
CI (Continuous Integration) Prozess zu definieren damit bei jeder Integration überprüft wird dass die
Software weiterhin funktioniert. Auch die LaTeX Dokumentation wird während dieser Zeit compiliert und überprüft.

\subsubsection{Vergleichstabelle}

\begin{table}[!h]
    \begin{tabular}{|l|l|l|ll}
    \cline{1-3}
    \textbf{CI System} & \textbf{Vorteile}                                                                       & \textbf{Nachteile}                                                                                              &  &  \\ \cline{1-3}
    Jenkins            & \begin{tabular}[c]{@{}l@{}}-Grosse Community\\ -Mächtig\end{tabular}                    & \begin{tabular}[c]{@{}l@{}}-Ressourcen verbrauch\\ -Komplexität\\ -Bedingte Integration github.com\end{tabular} &  &  \\ \cline{1-3}
    CircleCI           & \begin{tabular}[c]{@{}l@{}}-Schnell\\ -Leichtgewichtig\end{tabular}                     & \begin{tabular}[c]{@{}l@{}}-Beschränkte Funktionalität\\ -Bedingte Integration github.com\end{tabular}          &  &  \\ \cline{1-3}
    TravisCI           & \begin{tabular}[c]{@{}l@{}}-Einfach\\ -Schnell\\ -Integriert in github.com\end{tabular} & -Beschränkte Funktionalität                                                                                     &  &  \\ \cline{1-3}
    \end{tabular}
\end{table}

\subsubsection{Entscheidung}
Welches CI System eingesetzt wird ist nicht zentral für dieses Projekt. Da der Fokus auf der Funktionalität des Zuges Leichtgewichtig Leichtgewichtig
liegt und nicht auf den Entwicklungswerkzeugen. Trotzdem wollten wir eine fundierte Entscheidung treffen um das bestmögliche für diese Projekt auszuwäheln.
Schlussendlich basiert die Entscheidung auf der Integration mit github.com und der einfachheit in Kombination der Geschwindigkeit.
Deshalb wird TravisCI genutzt um den Source Code bzw. die Dokumentaion zu compilieren und den Source Code zu testen.

\end{document}