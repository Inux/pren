\documentclass[../../main.tex]{subfiles}
\begin{document}
In dieser Arbeit liegt das Gesamtkonzept für einen autonomen Schnellzug vor. Dieses Konzept wurde im HS18 im Rahmen des PREN1 entwickelt und wird im FS19 im Modul PREN2 realisiert. Die Aufgabe besteht darin, einen Schnellzug zu entwickeln, welcher eine definierte Strecke mit Geraden und Kurven so schnell wie möglich zurücklegt. Im Startbereich muss mittels einer am Zug befestigen Konstruktion ein Holzwürfel auf den Zug gehoben und transportiert werden. Danach muss der Zug eine Lichtschranke passieren und zwei Runden auf der Strecke so schnell wie möglich absolvieren. Dabei muss er ein seitlich befindendes Signal mit Nummer erkennen, welches die Halteposition signalisiert. Diese Halteposition muss in der dritten Runde angefahren werden, und der Zug soll dort so nahe wie möglich anhalten. Die erkannte Nummer soll auch mittels akustischem Signal ausgegeben werden. \\
Der Zug wurde mit folgenden Schwerpunkten entwickelt:
\begin{itemize}
    \item Kompaktheit
    \item Einfachheit
    \item niedriges Gewicht
    \item Robustheit
    \item niedrige Kosten
\end{itemize}
Im Hauptteil dieser Arbeit wird das Konzept des Zuges beschrieben. Die Beschreibung ist aufgeteilt in die drei
Fachgebiete Maschinentechnik, Elektrotechnik und Informatik. Im Maschinentechnik-Bereich wird die mechanische
Grundkonstruktion des Zuges und des Krans für den Holzwürfel erläutert. Der Antrieb, die Sensorik und die
Stromversorgung des Zuges wird im Elektrotechnikteil beschrieben. Im Abschnitt Informatik wird die Signalerkennung, die
akustische Ausgabe und der Softwareaufbau beschrieben. Projektmanagement, Kostenübersicht und Schlussdiskussion sind im hinteren Teil der Arbeit zu finden.\\
In diesem Konzept wurden bei der Entwicklung die Schwerpunkte Kompaktheit, Einfachheit und niedriges Gewicht berücksichtigt, um eine optimale maximale Geschwindigkeit mit dem Zug zu erreichen. Dabei soll aber auch ein Schwergewicht auf Robustheit und Prozesssicherheit gelegt werden. Zusätzliche Optimierungen, besonders in der maximalen Geschwindigkeit, können während dem PREN2, der Realisierungsphase, erzielt werden.
\pagebreak
\end{document}
