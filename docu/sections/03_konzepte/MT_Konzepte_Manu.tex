\documentclass[../../main.tex]{subfiles}
    
    \lstset{basicstyle=\small,
      showstringspaces=false,
      commentstyle=\color{black},
      keywordstyle=\color{blue}
    }

    \graphicspath{{images/}{../../images/Konzept Maschinentechnik/}}

    \begin{document}
    \subsection{Konzeptentwicklung Maschinentechnik}
    Im folgenden Kapitel wird die Konzeptentwicklung im Bereich Maschinentechnik aufgezeigt. Die Problemstellung ist in verschiedene Teilfunktionen unterteilt, welche in diesem Kapitel\\

    \textbf{Teilfunktionen}\\
    Die Objekterkennung kann je nach Problemstellung mit unterschiedlichen Disziplinen (Mechanik,
    Elektrotechnik und Informatik) gelöst werden. In der Tabelle \ref{tab:obj_disziplin} wird
    aufgelistet welche Problemstellung mit welcher Disziplin gelöst werden kann.
    %Tabelle mit DisziplinKonzepte
    \begin{flushleft}
        \begin{table}[h]
        \begin{tabular}{ | l | p{11cm} |}
        \hline
        \textbf{Disziplin} & \textbf{Konzepte} \\ \hline
        Mechanik & \begin{itemize}
                        \item Spurrichtungserkennung
                    \end{itemize} \\ \hline
        Elektrotechnik & \begin{itemize}
                             \item Würfelerkennung
                             \item Spurrichtungserkennung
                             \item Lichtraumprofilerkennung
                         \end{itemize} \\ \hline
        Informatik &  \begin{itemize}
                        \item Würfelerkennung
                        \item Spurrichtungserkennung
                        \item Lichtraumprofilerkennung
                        \item Signalerkennung
                      \end{itemize} \\ \hline
        \end{tabular}
        \caption{Disziplinentabelle}
        \label{tab:obj_disziplin}
    \end{table}
    \end{flushleft}

    Im Folgenden werden die möglichen Konzepte pro Problemstellung analysiert. Für die Disziplin Informatik
    wird immer das Konzept «Kamera» verwendet.

    \subsubsection{Würfelerkennung}
    Die Problemstellung «Würfelerkennung» kann entweder mit der Disziplin Elektrotechnik oder Informatik
    gelöst werden. Bei der «Würfelerkennung» muss im Startbereich einen Holzwürfel erkennt,
    Position des Würfels ermittelt und diesen mittels Kran auf den Zug geladen werden 
    (siehe Abbildung \ref{fig:wurfelerkennung}).

\end{document}

