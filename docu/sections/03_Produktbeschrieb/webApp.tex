\documentclass[../../main.tex]{subfiles}
\begin{document}
\subsection{Webapplikation}
In diesem Kapitel wird auf die Implementierung der technischen Komponente 'Web Applikation' eingegangen. Die Webapplikation wurde am Anfang des Entwicklungsprozess eingeführt und ist somit nicht in Pren1 dokumentiert. 
\subsubsection{Anforderung}

Ziel und Zweck der Applikation ist es eine zentrale Übersicht der einzelnen Module / Komponente zu erhalten und diese zu testen. Der Zug wird mittels Startknopf über das Webinterface gestartet.

\paragraph{Anforderungen}
\begin{itemize}
    \item Übersicht
      \subitem Heartbeat
      \subitem Informationen zu Zug \& Fahrt
    \item Module testen
    \item Fahrt starten
\end{itemize}

\subsubsection{Lösung}
Der Client ist eine klassische Kombination aus html \& javascript. Dieser spricht über eine Rest API mit unserem Server. Auf dem Server verwenden wir Sanic, eine auf Python basierte Servertechnologie.

\paragraph{Technologien / Komponente}
 

\paragraph{Aufbau}
\dirtree{%
.1 /webapp.
.2 /static.
.3 /Resources.
.3 /Scripts.
.3 /Style.
.2 index.html.
.2 server.py.
}



\paragraph{Sanic}
Sanic ist ein Python 3.6+ Webserver und Webframework, welches den Fokus auf Performanz legt. Durch den Einsatz der im Python 3.5 hinzugefügten 'async/await' Syntax wird der Code nicht-blockierend und schneller.

Das Projekt ist für kleine Projekte ideal, auch ist Python ein grosser Pluspunkt, weshalb wir uns für diese Technologie entschieden haben.

\paragraph{REST API}

\paragraph{Soll-Ist-Vergleich}

\subsubsection{Entwicklungsablauf}

\subsubsection{Testing}

\subsubsection{Reflexion}

\end{document}