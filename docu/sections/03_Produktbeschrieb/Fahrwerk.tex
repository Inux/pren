\documentclass[../../main.tex]{subfiles}

\graphicspath{{images/Maschinentechnik/}{../../images/Maschinentechnik/}}

\lstset{basicstyle=\small,
      showstringspaces=false,
      commentstyle=\color{black},
      keywordstyle=\color{blue}
    }

\begin{document}

% Produktbeschreibung des Funktionsmusters\\
% Produktbeschreibung\\
% Vergleich Theorie und Endprodukt\\
% Ergebnisse von Versuchen und Tests\\

\subsection{Fahrwerk} \label{mt_Fahrwerk}

Das Fahrwerk (siehe Abbildung \ref{fig:konzeptfahrwerk}) des Zuges besteht aus einem Antriebswagen (Position 1), einem Führungswagen (Position 2) und der Ladefläche (Position 3), auf welcher die Würfelaufnahme angebracht ist. Die drei Positionen werden in den nächsten Abschnitten genauer vorgestellt. Die Position 4 ist das Ladegut, der Holzwürfel.\\

\begin{figure}[H]
   \centering
   \includegraphics[width=.95\textwidth]{../../images/Maschinentechnik/lokomotive.PNG}
   \includegraphics[width=.95\textwidth]{../../images/Maschinentechnik/lokomotive2.PNG}
   \caption {Konzept und Ergebnis des Fahrwerkes}
   \label{fig:konzeptfahrwerk}
\end{figure}

\textbf{Antriebswagen}\\
Der Antriebswagen (siehe Abbildung \ref{fig:antriebswagen1} bzw. \ref{fig:antriebswagen2}) besteht aus zwei gelagerten Achsen, welche beide durch einen Riemen angetrieben werden. Der Motor treibt ein kleines Getriebe mit einem Übersetzungsverhältnis von 2:1 den Riemen und somit die beiden Achsen an. Der Motor befindet sich in der Mitte des Riemens, mit dem Ziel den Schwerpunkt möglichst tief zu halten. Alle Räder sind mit Schrauben und Unterlagscheiben an den Achsen befestigt, damit ein schneller und einfacher Radwechsel möglich ist. Zwischen den Rädern befindet sich auf beiden Seiten des Wagens je ein Schleifkontakt aus Federblech, welcher mit einer kleinen Vorspannung auf die Gleise drückt. So werden kleine Unebenheiten auf der Strecke kein Problem für die Stromabnahme. An der Spitze des Wagens befinden sich zwei Kamerhalterungen, welche einstellbar und mit einem Gummipuffer gedämpft sind. Durch einen Bügel und einer Radiallagerung ist der Antriebswagen mit dem Ladungsträger verbunden.\\

\begin{figure}[H]
\begin{minipage}{.6\textwidth}
  \includegraphics[width=.9\textwidth]{antriebswagen.PNG}
   \caption[]{Konzept Antriebswagen}
   \label{fig:antriebswagen1}
 \end{minipage}
\begin{minipage}{.4\textwidth}
\begin{table}[H] \centering
  \begin{tabular}{|l|l|}
  \hline
  \textbf{Position} & \textbf{Bezeichnung}\\
  \hline
  Position 1          & Wagen\\
   \hline
  Position 2          & Antriebseinheit\\
  \hline
  Position 3          & Kameras\\
  \hline
  Position 4          & Zahnriemen\\
  \hline
 \end{tabular}
 \caption{Positionsnummern}
 \label{tab:expl_antriebswagen}
 \end{table}
 \end{minipage}
\end{figure}

\begin{figure}[H]
    \centering
    \includegraphics[width=0.495\textwidth]{../../images/Maschinentechnik/antriebswagen3.PNG}
    \includegraphics[width=0.455\textwidth]{../../images/Maschinentechnik/antriebswagen4.PNG}
    \caption {Antriebswagen Endprodukt}
    \label{fig:antriebswagen2}
\end{figure}

Direkt unterhalb der Kamerahalterung befindet sich eine Kurvenvorrichtung (siehe Abbildung \ref{fig:kurvenvorrichtung}), welche die Gefahr des Entgleisens des Zuges minimiert. Sie funktioniert folgendermassen: Durch zwei Stahlstifte, welche radial die Kontur des Gleises haben und drehbar gelagert sind wird die Lokomotive in der Spur geführt. Die beiden Federn sorgen für den nötige Anpressdruck, damit die Vorrichtung in die Gleise verkeilt wird.\\

\begin{figure}[H]
  \centering
  \includegraphics[width=0.37\textwidth]{../../images/Maschinentechnik/kurvenvorrichtung.PNG}
  \includegraphics[width=0.4\textwidth]{../../images/Maschinentechnik/testkurvenvorrichtung2.PNG}
  \caption {Kurvenvorrichtung in unmontiertem und montiertem Zusatand}
  \label{fig:kurvenvorrichtung}
\end{figure}

\pagebreak

\textbf{Maximale Beschleunigung}\\
Der Motor kann gemäss Datenblatt ein maximales Moment von 85.6 mNm erreicht werden. Durch das Übersetzungsverhältnis 2:1 des Getriebes wird ein Moment von 171.2 mNm auf die beiden Radachsen übertragen und somit eine theoretische maximale beschleunigung von 44.79 Meter pro Sekunde im Quadrat erreichen (siehe Tabelle \ref{tab:groessen_beschleunigung}):

$$F_{Rad}=\frac{F_{G}}{8}=\frac{3kg \cdot 9.81m/s^2}{8}=0.375N$$

$$F_{Reibung}=F_{Rad} \cdot k=0.375N \cdot 0.3=0.1125N$$

$$M_{Rad}=M_{Getriebe} \cdot 0.5= 171.2 mNm \cdot 0.5 = 85.6 mNm$$

$$M_{Rad}=F_{Reibung} \cdot a_{max, Getrtiebe} \cdot D_{Rad,Endprodukt} \cdot 0.5$$

$$\boldsymbol{a_{max, Getriebe}}=\frac{M_{Rad}}{\frac{F_{Reibung}}{g} \cdot D_{Rad} \cdot 0.5}=\frac{0.0856Nm}{\frac{0.1125N}{9.81m/s^2} \cdot 0.026m \cdot 0.5}=\boldsymbol{58.53m/s^2}$$\\

\begin{table}[H] \centering
  \begin{tabular}{|l|l|}
  \hline
  \textbf{Grösse} & \textbf{Wert}\\
  \hline
  Durchmesser Rad (Konzept) [D,Konzept]          & 22 Millimeter\\
   \hline
   Durchmesser Rad (Endprodukt) [D,Endprodukt]          & 26 Millimeter\\
   \hline
  Reibungskoeffizient [k]      & 0.3\\
  \hline
  \end{tabular}

  \caption{Grössen für die Beschleunigungsberechnung}
  \label{tab:groessen_beschleunigung}
  \end{table}

Das theoretische maximale Moment beziehungswiese die maximale Drehzahl kann jedoch nicht erreicht werden da diese durch die vorgegebene Speisung beschränkt ist.\\
In der Konzeptphase wurde die theoretische Maximalbeschleunigung gemäss CAD-Daten der Lokomotive (siehe Tabelle \ref{tab:groessen_beschleunigung}) wie folgt berechnet:

$$M_{Rad,Konzept}=F_{Rad} \cdot 0.5 \cdot D_{Rad_Konzept}= 0.1125N \cdot 0.5 \cdot 22mm = 1.24mNm$$
$$M_{Rad,Prototyp}=F_{Rad} \cdot 0.5 \cdot D_{Rad,Endprodukt}= 0.1125N \cdot 0.5 \cdot 26mm = 1.46mNm$$

$$\boldsymbol{a_{max}}=\frac{F_{Reibung}}{\frac{F_{Rad}}{g}}=\frac{0.1125N}{\frac{0.375N}{9.81m/s^2}}=\boldsymbol{2.94m/s^2}$$
\\

\pagebreak

\textbf{Maximale Geschwindigkeit}\\
Die Maximale Geschwindigkeit in der Kurve wurde in der Theorie mit den gegebenen Grössen (siehe Tabelle \ref{tab:geschwindigkeitsberechnung}) folgendermassen berechnet:\\

\begin{table}[H] \centering
    \begin{tabular}{|l|l|}
    \hline
    \textbf{Grösse} & \textbf{Wert}\\
    \hline
    Minimaler Radius  [r]                               & 0.8 Meter\\
     \hline
    Masse [m]                                           & 3 Kilogramm\\
    \hline
    Schwerpunkt in x-Achse (maximaler Wert) [x]         & 0.0225 Meter\\
    \hline
    Schwerpunkt in y-Achse (maximaler Wert) [y]         & 0.05 Meter\\
    \hline
    \end{tabular}

    \caption{Grössen für die Geschwindigkeitsberechnung}
    \label{tab:geschwindigkeitsberechnung}
    \end{table}

Die Gewichts- und Zentripetalkraft, welche das Gleichungssystem für die Geschwindigkeitsberechnung bilden, sind wie folgt definiert:

$$F_{G}=m \cdot g=3kg \cdot 9.81m/s^2=29.4N$$

$$F_{max, z}=\frac{F_{G} \cdot x}{y}=\frac{29.4N \cdot 0.0225m}{0.05m}=13.24N$$

\begin{figure}[H]
    \centering
    \includegraphics[width=0.55\textwidth]{schwerpunkt.PNG}
    \caption {Schwerpunkt der Lokomotive}
    \label{fig:schwerpunkt}
\end{figure}

Da das Drehmoment eine vektorielle Grösse ist, müssen die beiden entstehenden Momente am Drehpunkt ''P'' am Gleis zusammen Null ergeben (siehe Abbildung \ref{fig:schwerpunkt}). Oder anders gesagt, müssen die beiden Momente gleich gross sein, damit das System ''statisch'' bestimmt ist. Die Berechnungen sind auf den kleinsten Kurvenradius ausgelegt, da dort die grösste Zentripetalkraft entsteht. Somit ergibt sich eine maximale Geschwindigkeit von 1.53 Meter pro Sekunde:

$$F_{max, z}=\frac{m \cdot v_{max}^2}{r};\boldsymbol{v_{max}}=\sqrt\frac{F_{max, z}\cdot r}{m}=\sqrt\frac{13.24N \cdot 0.8m}{3kg}=\boldsymbol{1.53m/s}$$\\

\begin{figure}[H]
  \begin{minipage}{.5\textwidth}
    \textbf{Führungwagen}\\
    Der Führungswagen in Abbildung \ref{fig:fuehrungswagen1} ist grundsätzlich gleich aufgebaut wie der Antriebswagen. Er hat zwei gelagerte Achsen, an welchen je zwei Räder mit Schrauben und Unterlagscheiben befestigt sind. Zwischen den Rädern hat es jeweils einen Schleifkontakt aus Federstahl, welcher auf die Gleise vorgespannt ist und den Strom somit an einer zweiten Stelle vom Gleis abnimmt. Am hinteren Ende des Führungswagen befindet sich wiederum die Vorrichtung für die Kurvenfahrt. Durch einen Stift und ein Radiallager ist der Ladungsträger mit dem Führungswagen schwenkbar verbunden.
   \end{minipage}
  \begin{minipage}{.5\textwidth}
    \flushright
    \includegraphics[width=0.95\textwidth]{fuehrungswagen2.JPG}
    \caption {Führungswagen}
    \label{fig:fuehrungswagen1}
    \end{minipage}
\end{figure}

Der Aufbau des Führungswagens ist wie konzipiert, mit Ausnahme der Räder, finalisiert und hergestellt worden. Der Führungswagen ist folgendermassen aufgebaut:\\

\begin{figure}[H]
  \centering
  \includegraphics[width=0.7\textwidth]{fuehrungswagen.PNG}
  \caption {Konzept des Führungswagens}
  \label{fig:fuehrungswagen2}
\end{figure}

\begin{table}[H] \centering
  \begin{tabular}{|l|l|}
  \hline
  \textbf{Position} & \textbf{Bezeichnung}\\
  \hline
  Position 1          & Drehachse Wagen-Ladungsträger (eingepresst)\\
   \hline
  Position 2          & Achsen (Gewinde an beiden Seiten, Anfräsfläche für Gabelschlüssel)\\
   \hline
  Position 3          & Sicherungsring für Rillenkugellager\\
  \hline
  Position 4          & Eingepresstes Rillenkugellager (Festlager) bzw. Loslager\\
  \hline
  Position 5          & Rad\\
  \hline
  Position 6          & Unterlagscheibe\\
  \hline
  Position 7          & Zylinderschraube\\
  \hline
  \end{tabular}
\caption{Positionsnummern}
\label{tab:expl_antriebswagen}
\end{table}

\pagebreak

\textbf{Befestigung der Elektronik Komponenten}\\
Der Montageraum für die Elektrokomponenten ist beim Endprodukt nicht nur auf dem Führungswagen, sondern ebenfalls auf dem Antriebswagen. Auf dem Antriebswagen sind Elektronik Komponenten mit Klett an der Antriebseinheit befestigt. Weitere wie Buzzer oder Näherungssensor an den Freien Flächen des Zuges angebracht. Auf der Abbildung \ref{fig:elektrokomponenten} sind die montierten Elektronik Komponenten und deren Verkablung ersichtlich. Die Komponenten sind detaillierter in Kapitel \ref{et_verbindungen} beschreiben.\\

\begin{figure}[H]
  \centering
  \includegraphics[width=1\textwidth]{lokomotive3.PNG}
  \caption {Elektrokomponenten}
  \label{fig:elektrokomponenten}
\end{figure}

\begin{figure}[H]
  \begin{minipage}{.5\textwidth}
    \textbf{Stromabnahme}\\
    Die anfänglich konzipierten Schleifkontakte (siehe Abbildung \ref{fig:schleifkontakte}) gebogen aus Federstahl erweisen sich als eine einfache und zuverlässige Art der Stromabnahme. Für die ersten Testfahrten wurde eine Stromabnahme am hinteren Wagen, dem Führungswagen, angebracht. Es wurde jedoch festgestellt, wenn Unebenheiten der Strecke vorhanden sind, die Möglichkeit eines Kontaktverlustes besteht. Somit wurde in einem zweiten Schritt eine zweite Stromabnahme am Antriebswagen befestigt, welche diese Gefahr minimiert.\\
   \end{minipage}
  \begin{minipage}{.5\textwidth}
    \flushright
    \includegraphics[width=0.9\textwidth]{schleifkontakt.PNG}
    \caption {Schleifkontakte}
    \label{fig:schleifkontakte}
    \end{minipage}
\end{figure}

\end{document}