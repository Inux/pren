\documentclass[../../main.tex]{subfiles}

\graphicspath{{images/Steuerungssoftware/}{../../images/Steuerungssoftware/}}
\begin{document}
\subsection{Steuerungssoftware}
In diesem Kapitel wird auf die Implementierung der technischen Komponente 'Web Applikation' eingegangen. Die Webapplikation wurde am Anfang des Entwicklungsprozess eingeführt und ist somit nicht in Pren1 dokumentiert. 
\subsubsection{Anforderung}

Ziel und Zweck der Applikation ist es eine zentrale Übersicht der einzelnen Module / Komponente zu erhalten und diese zu testen. Somit fungiert die WebApp als Testing/Monitoring Komponente und ist für den Ablauf des Controlflows irrelevant. Der Zug kann mittels Startknopf über das Webinterface gestartet werden.

\paragraph{Anforderungen}
\begin{itemize}
    \item Übersicht
      \subitem Heartbeat
      \subitem Informationen zu Zug \& Fahrt
    \item Module testen
    \item Fahrt starten
\end{itemize}

\subsubsection{Lösung}
Der Client ist eine klassische Kombination aus html \& javascript. Dieser spricht über eine Rest API mit unserem Server. Auf dem Server verwenden wir Sanic, eine auf Python basierte Servertechnologie.

\paragraph{Technologien / Aufbau}

\textbf{Protobuf Nachrichten} \\
\textbf{Kommunikationsmodule} \\
\textbf{ZeroMQ Kommunikationsdiagramm} \\
\textbf{Ablauf (mit Code)} \\
\textbf{Codestruktur (Basisklasse)} \\

\subsubsection{Entwicklungsablauf}
Enwurf, sowie erster Prototyp der WebApp wurden von Steve Ineichen anfangs von PREN1 erstellt. Das Hinzufügen immer weiterer Komponenten führte dazu, dass die WebApp im Laufe der Entwicklung kontinuierlich erweitert wurde.

\subsubsection{Testing}
Eine spezifische Testingstrategie für Client oder Server existiert nicht. Die WebApp fungiert als Testingkomponente und ist somit Teststrategie der meisten Komponenten auf dem RasperryPi 3+.
\\
Das Testing erfolgt während des Entwicklungsablaufs automatisch. Wird eine neu eingebundene Komponente auf dem Client simuliert, so wird die Funktionalität der WebApp mitgetestet.

\subsubsection{Reflexion}
Eine Testingkomponente einzuführen war eine tolle Idee und hat sich während der Entwicklung sehr bewährt. Die WebApp liefert eine attraktive Übersicht betreffend den verschiedenen Komponenten.

\end{document}