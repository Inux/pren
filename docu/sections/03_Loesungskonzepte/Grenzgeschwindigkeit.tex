\documentclass[../../main.tex]{subfiles}
\begin{document}
\subsection{Grenzgeschwindigkeit}
Die Grenzgeschwindigkeit ist nicht nur abhängig von der Masse des Zuges, des Schwerpunktes, der Leistung des Antriebs und den Kurven Radien.
Es ist zentral, dass die Bilderkennung für die Signalerkennung und ebenfalls für die Gleiserkennung noch funktioniert.
Deshalb werden in diesem Kapitel die verschiedenen Grenzgeschwindigkeiten aufgelistet die näher betrachtet wurden.
Daraus kann der Flaschenhals bezüglich Grenzgeschwindigkeit herausgelesen werden und dementsprechend auch die Risiken
bestimmt werden.

\subsubsection{Auflistung}
Nachfolgend werden die einzelnen Grenzgeschwindigkeiten die berechnet wurden aufgelistet
\begin{itemize} %Die verschiedenen Grenzgeschwindigkeiten
    \item Grenzgeschwindigkeit ohne Bildverarbeitung
    \item Grenzgeschwindigkeit mit Signalerkennung
    \item Grenzgeschwindigkeit mit Signalerkennung \& Gleiserkennung
\end{itemize}

\textbf{Grenzgeschwindigkeit ohne Bildverarbeitung} \\
Die Grenzgeschwindigkeit ohne Bildverarbeitung basiert auf der Masse des Zuges, des Schwerpunktes, der Leistung des Motors und den Kurven Radien.
Alle Berechnungen zu diesem Thema sind in den Kapitel \ref{GeschwindigkeitsberechnungFahrwerk}.

\textbf{Grenzgeschwindigkeit mit Signalerkennung} \\
Die Grenzgeschwindigkeit mit Signalerkennung basiert auf der Grenzgeschwindigkeit ohne Bildverarbeitung. Zusätzlich wird aber berücksichtig wie lange eine
Aufnahme eines einzelnen Bildes dauert. Somit kann gewährleistet werden ob noch ein verwendbares Bild aufgenommen werden kann. Die eigentliche Bildanalyse
wird dabei vernachlässigt weil dies nicht während den Highspeed Runden passiert sondern anschliessend bei der Fahrt zum Haltesignal.

\textbf{Grenzgeschwindigkeit mit Signalerkennung \& Gleiserkennung} \\
Die Grenzgeschwindigkeit mit Signalerkennung \& Gleiserkennung basiert auf der Grenzgeschwindigkeit mit Signalerkennung. Dabei wird aber die Erkennung des Gleises
mit berücksichtigt. Die Gleiserkennung dient dazu Kurven zu erkennen und bietet somit einen weiteren Parameter für die Regelung.

\end{document}