\documentclass[../../main.tex]{subfiles}
\begin{document}
\subsection{Grenzgeschwindigkeit}
Die Grenzgeschwindigkeit ist nicht nur abhängig von der Masse des Zuges, des Schwerpunktes, der Leistung des Antriebs und den Kurven Radien.
Es ist zentral, dass die Bilderkennung für die Signalerkennung und ebenfalls für die Gleiserkennung noch funktioniert.
Deshalb werden in diesem Kapitel die verschiedenen Grenzgeschwindigkeiten aufgelistet, die näher betrachtet wurden.
Daraus kann der Flaschenhals bezüglich Grenzgeschwindigkeit herausgelesen und dementsprechend auch die Risiken
bestimmt werden.

\subsubsection{Auflistung}
Nachfolgend werden die einzelnen Grenzgeschwindigkeiten, die berechnet wurden, kurz beschrieben. Anschliessend wird in
einem Fazit erläutert wo der Flaschenhals bezüglich der Geschwindigkeit liegt.

\textbf{Grenzgeschwindigkeit ohne Bildverarbeitung} \\
Die Grenzgeschwindigkeit ohne Bildverarbeitung basiert auf der Masse des Zuges, des Schwerpunktes, der Leistung des Motors und den Kurven Radien.
Alle Berechnungen zu diesem Thema sind im Kapitel \ref{GeschwindigkeitsberechnungFahrwerk}.

Gemässe den Berechnungen handelt es sich um 1.28m/s beim engsten Kurvenradius von 0.8m . Geradeaus kann wesentlich mehr
erreicht werden. Wenn wir eine Gerade von 3m annehmen was wohl der maximal länge einer Geraden in PREN2 entspricht
annehmen erreichen wir maximal 3m/s.

\textbf{Grenzgeschwindigkeit mit Signalerkennung} \\
Basierend auf der Grenzgeschwindigkeit ohne Bildverarbeitung, wird zusätzlich berücksichtigt, wie lange eine
Aufnahme eines einzelnen Bildes dauert und wieviel Bilder gemacht werden müssen um die Umgebung genügend Abzudecken.
Somit kann gewährleistet werden, dass ein verwendbares Bild aufgenommen werden kann während den Hochgeschwindigkeits-Runden. Die eigentliche Bildanalyse
wird dabei vernachlässigt, weil dies nicht während den Hochgeschwindigkeits-Runden passiert, sondern anschliessend bei
der Fahrt zum Haltesignal.

Maximal Geschwindigkeit: 3m/s
Umgebung Abdeckung: 10cm (des Weges)

Pi Bildaufnahme: 20ms
Bilder Pro Meter: 3m/s * 0.02s = 0.06m = 6cm

Die normale Bildaufnahme ist also genügend schnell, deckt alle Bereiche ab da alle 6cm ein Bild gemacht werden kann bei
der maximalen Geschwindigkeit und ist somit kein Hindernis für eine rasche Fahrt.

\textbf{Grenzgeschwindigkeit mit Signalerkennung \& Gleiserkennung} \\
Zusätzlich zu den anderen beiden Grenzgeschwindigkeiten wird nun noch die Gleiserkennung berücksichtigt welche gemessen
etwa 100ms
pro Bild benötigt. Die Gleiserkennung dient dazu, Kurven zu erkennen und bietet somit einen weiteren Parameter für die
Regelung um eine maximale Geschwindigkeit zu fahren. Die Kamera kann das Gleis mindestens 20cm im voraus genügend gut
erkennen. Dies ist wie vorhin gezeigt wichtig bei der Berechnung.

Maximal Geschwindigkeit: 3m/s
Umgebung Abdeckung: 20cm (des Weges)

Pi Bildaufnahme: 100ms
Bilder Pro Meter: 3m/s * 0.1s = 0.1m = 10cm

Die Gleiserkennung ist somit wie die Signalerkennung genügend schnell, um auch mit der maximalen Geschwindigkeit von 3m/s
zu fahren.

\subsubsection{Fazit}
Die verschiedenen Grenzgeschwindigkeiten wurden analysiert und das Ergebnis ist klar. Der vermutliche Flaschenhals liegt
bei der vorgegeben Leistung von 60 Watt und der daraus resultierenden Kombination von Motor \& Mechanik. Dennoch ist es bemerkenswert
was für Geschwindigkeiten möglich sind. Ob wir diese auch erreichen wird sich in PREN2 zeigen, wir sind aber sicher eine
gute Konstruktion, ein gutes elektrisches Design und passende Softwaremittel gewählt zu haben.

\end{document}