\documentclass[../../main.tex]{subfiles}
\begin{document}
\subsection{Grenzgeschwindigkeit}
Die Grenzgeschwindikeit ist nicht nur abhängig von der Masse des Zuges, des Schwerpunktes, der Leistung des Antriebs und den Kurven Radien.
Es ist zentral, dass die Bilderkennung ebenfalls noch funktioniert. Deshalb werden in diesem Kapitel die verschiedenen Grenzgeschwindigkeiten aufgelistet.
Daraus kann der Flaschenhals bezüglich Grenzgeschwindikeit herausgelesen werden.

\subsubsection{Auflistung}
Nachfolgend werden die einzelen Grenzgeschwindigkeiten die berechnet wurden aufgelistet
\begin{itemize} %Die verschiedenen Grenzgeschwindigkeiten
    \item Grenzgeschwindikeit ohne Bildverarbeitung
    \item Grenzgeschwindikeit mit Signalerkennung
    \item Grenzgeschwindikeit mit Signalerkennung \& Gleiserkennung
\end{itemize}

\textbf{Grenzgeschwindikeit ohne Bildverarbeitung} \\
Die Grenzgeschwindikeit ohne Bildverarbeitung basiert auf der Masse des Zuges, des Schwerpunktes, der Leistung des Motors und den Kurven Radien.

\textbf{Grenzgeschwindikeit mit Signalerkennung} \\
Die Grenzgeschwindikeit mit Signalerkennung basiert auf der Grenzgeschwindikeit ohne Bildverarbeitung. Zusätlich wird aber berücksichtig wie lange eine
Aufnahme eines einzelnen Bildes dauert. Somit kann gewährleistet werden ob noch ein verwendbares Bild aufgenommen werden kann. Die eigentliche Bildanalyse
wird dabei vernachlässigt weil dies nicht während den Highspeed Runden passiert sondern anschliessend bei der Fahrt zum Haltesignal.

\textbf{Grenzgeschwindikeit mit Signalerkennung \& Gleiserkennung} \\
Die Grenzgeschwindikeit mit Signalerkennung \& Gleiserkennung basiert auf der Grenzgeschwindikeit mit Signalerkennung. Dabei wird aber die Erkennung des Gleises
mit berücksichtigt. Die Gleiserkennung dient dazu Kurven zu erkennen und bietet somit einen weiteren Parameter für die Regelung.

\end{document}