\documentclass[../../main.tex]{subfiles}
\begin{document}
Damit das Schienenfahrzeug alle Teilaufgaben optimal erfüllen kann, soll im Verlauf von PREN1 ein Lösungskonzept entwickelt und gewisse Teilfunktionen getestet werden. Das Gesamtkonzept besteht aus verschiedenen Teilkonzepten, für die jeweils mehrere Lösungsvorschläge aufbereitet und anschliessend bewertet wurden. Der Entscheidungsprozess wurde im Rahmen eines Testates dokumentiert und wird in diesem Dokument nicht thematisiert (befindet sich im Anhang). In den folgenden Unterkapiteln wird jede definierte Teilfunktion des Schienenfahrzeugs beschrieben und die jeweilige Lösung dazu präsentiert. Liegen bereits praktische Tests vor, werden diese ebenfalls dargestellt. Für alle Probleme soll eine möglichst einfache und doch effektive Lösungsvariante präsentiert werden. Im unmittelbar nächsten Kapital wird der Ablauf mit einem Zustandsdiagramm dargestellt.\\

\textbf{Übersicht Lösungskonzepte}
\begin{itemize}
    \item Ablauf
    \item Fahrwerk
    \item Würfelaufnahme / Transport
    \item Motorauslegung
    \item Akustik
    \item Beschleunigung / Geschwindigkeit
    \item Elektronik \& Komponenten
    \item Signalerkennung \& Gleiserkennung
    \item Zusammenspiel RPI und Tiny
    \item Software
\end{itemize}
\end{document}