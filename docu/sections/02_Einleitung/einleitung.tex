\documentclass[../../main.tex]{subfiles}
\begin{document}
\subsection{Einleitung}
In dieser Arbeit liegt das Gesamtkonzept für einen autonomen Schnellzug vor. Dieses Konzept wurde im HS18 im Rahmen des PREN1 entwickelt und im FS19 in PREN2 realisiert. Die Aufgabe besteht darin einen Schnellzug zu entwickeln, welcher so schnell wie möglich eine definierte Strecke mit Geraden und Kurven so schnell wie möglich zurücklegt. Im Startbereich muss der Zug mittels am Zug befestigen Konstruktion einen Holzwürfel auf den Zug transportieren. Danach muss der Zug durch eine Lichtschranke die Strecke in zwei Runden absolvieren. Dabei muss er ein seitlich befindendes Signal mit Nummer erkennen, welche die Halteposition verrät. Diese Halteposition muss dann in der dritten Runde angefahren werden und so nahe wie möglich angehalten werden. Die erkannte Nummer soll auch mittels akustischen Signal ausgegeben werden. Der Zug wurde mit folgenden Schwerpunkten entwickelt:
\begin{itemize}
    \item Kompaktheit
    \item Einfachheit
    \item Gewicht
    \item Robustheit
    \item Kosten
\end{itemize}
Im Hauptteil dieser Arbeit wird das Konzept des Zuges beschrieben. Die Beschreibung ist aufgeteilt in die drei
Fachgebiete Maschinentechnik, Elektrotechnik und Informatik. Im Maschinentechnik Bereich wird die mechanische
Grundkonstruktion des Zuges und des Krans für den Holzwürfel erläutert. Der Antrieb, die Sensorik und die
Stromversorgung des Zuges wird im Elektrotechnikteil beschrieben. Im Abschnitt Informatik wird die Signalerkennung, die
akustische Ausgabe und der Softwareaufbau beschrieben. Berechnung für die erwartete maximale Geschwindigkeit und die
durchgeführten Versuchsaufbauten sind im hinteren Teil der Arbeit beschrieben.\\
In diesem Konzept wurde wie oben erwähnt mit Schwerpunkten wie Kompaktheit, Einfachheit und niedrigem Gewicht entwickelt um eine gute maximale Geschwindigkeit mit dem Zug zu erreichen. Dabei soll aber auch ein Schwergewicht auf Robustheit und Prozesssicherheit gelegt werden. Zusätzliche Optimierungen besonders in der maximalen Geschwindigkeit können während dem PREN2, der Realisierungsphase, erzielt werden.
\pagebreak
\end{document}