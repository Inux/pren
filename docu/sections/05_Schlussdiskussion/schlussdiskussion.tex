\documentclass[../../main.tex]{subfiles}
\begin{document}

\subsection{Kosten}
Gemäss der Aufgabenstellung steht ein Budget von 500 Franken zur Verfügung. In Tabelle \ref{tab:kosten_total} wurden alle Komponenten in einer Kostenübersicht zusammengetragen. Gemäss dieser Übersicht gibt es noch eine Reserve von ca. 50 Franken für nicht vorhergesehene Änderungen.\\
In dieser Übersicht nicht enthalten sind Komponenten, welche aus der Mechanikwerkstatt oder dem Elektronik Labor der
HSLU bezogen werden können. Auch sind Stundenansätze für Werkstattpersonal oder 3D-Drucker nicht berücksichtigt. \\

\begin{table}[H] \centering
    \begin{tabular}{|p{6cm}|l|r|r|r|}
    \hline
    \textbf{Beschreibung}                                   & \textbf{Lieferant} & \textbf{Anzahl} & \textbf{Stückpreis} & \textbf{Gesammtpreis} \\ \hline
    Raspberry Pi 3 Model B+                                 & pi-shop            & 1               & CHF 39.00           & CHF 39.00             \\ \hline \nocite{PiShopPi3ModelBp}
    Raspberry Pi Zero W                                     & pi-shop            & 1               & CHF 10.80           & CHF 10.80             \\ \hline \nocite{PiShopPiZero}
    Raspberry Pi Kamera Beleuchtung                         & pi-shop            & 2               & CHF 24.90           & CHF 49.80             \\ \hline \nocite{PiShopBrightPi}
    Raspberry Pi Kamera                                     & pi-shop            & 2               & CHF 32.90           & CHF 65.80             \\ \hline \nocite{PiShopPiKamera}
    Passiver Buzzer                                         & play-zone          & 1               & CHF 5.00            & CHF 5.00              \\ \hline \nocite{PlayZonePassiverBuzzer}
    StromPi 3                                               & Reichelt           & 1               & CHF 41.64           & CHF 41.64             \\ \hline \nocite{ReicheltStromPi}
    StromPi 3 Battery Pack, 1000 mAh                        & Reichelt           & 1               & CHF 31.20           & CHF 31.20             \\ \hline \nocite{ReicheltStromPiAkku}
    Beschleunigungssensor ADXL345                           & aliexpress         & 1               & CHF 1.00            & CHF 1.00              \\ \hline \nocite{AliExpressADXL345}
    Tiny K22                                                & hslu               & 1               & CHF 26.00           & CHF 26.00             \\ \hline
    Arduino IBT\_2 (DC-Motoren Treiber) (43A)               & wish               & 1               & CHF 9.00            & CHF 9.00              \\ \hline \nocite{WisIBT2}
    DC\-DC CC Converter (12A)                               & wish               & 1               & CHF 3.00            & CHF 3.00              \\ \hline \nocite{WishDCDCConverter}
    Ultraschallsensor HS-SR04                               & aliexpress         & 1               & CHF 0.77            & CHF 0.77              \\ \hline \nocite{AliExpressHCSR04}
    TOF Sensor VL53L0X                                      & aliexpress         & 1               & CHF 3.69            & CHF 3.69              \\ \hline \nocite{AliExpressVL53L0X}
    LM2596 DC-DC Schritt-down Converter (2.5A)              & aliexpress         & 1               & CHF 0.90            & CHF 0.90              \\ \hline \nocite{AliExpressLM2596}
    Strommesswiderstand 13FR200E                            & Distrelec          & 1               & CHF 2.80            & CHF 2.80              \\ \hline \nocite{Distrelec13FR200E}
    Logic\-Level\-Converter 5V zu 3,3V                      & aliexpress         & 1               & CHF 0.90            & CHF 0.90              \\ \hline \nocite{AliExpress5V33VConverter}
    Tower Pro Micro Servo SG90                              & wish               & 1               & CHF 1.60            & CHF 1.60              \\ \hline \nocite{WishTowerpro48}
    Mini Modul PWM Speed Control Über L298N                 & aliexpress         & 1               & CHF 0.69            & CHF 0.69              \\ \hline \nocite{WishTowerpro48}
    Antriebsmotor mit Encoder                               & Sponsor            & 1               & CHF 30.00           & CHF 30.00             \\ \hline
    Schwenker Motor mit Encoder                             & Sponsor            & 1               & CHF 25.00           & CHF 25.00             \\ \hline
    Zahnriehmen                                             & Mädler             & 1               & CHF 15.00           & CHF 15.00             \\ \hline \nocite{MadlerZahnriemen}
    Zahnriehmenrad                                          & Mädler             & 3               & CHF 8.00            & CHF 24.00             \\ \hline \nocite{MadlerZahnriemenrad}
    Kupplungen                                              & Mädler             & 1               & CHF 30.00           & CHF 30.00             \\ \hline \nocite{MadlerKupplung}
    Zahnräder                                               & Mädler             & 3               & CHF 10.00           & CHF 30.00             \\ \hline \hline \nocite{MadlerZahnrad}
    \textbf{Total}                                          & \textbf{}          & \textbf{30}     & \textbf{}           & \textbf{CHF 447.59}   \\ \hline
    \end{tabular}
    \caption{Kostenübersicht Gesamtkonzept}
    \label{tab:kosten_total}
    \end{table}

\pagebreak

\subsection{Risikomanagement}
Während der Konzeptionsphase, in welcher dieses Konzept entstanden ist, wurden in bestimmten zeitlichen Intervallen Risikoanalysen durchgeführt. Diese flossen in die Entscheidungsfindung für alle Lösungskonzepte mit ein. Zum Schluss der Konzeptphase, anhand des fertigen Konzeptes, wurde nochmals eine Risikoanalyse durchgeführt pro Disziplin. Diese Risikoanalyse wird für das PREN2 entscheidend sein, um die geplanten Lösungskonzepte erfolgreich realisieren zu können.

    \begin{longtable}{|l|p{2.5cm}|p{3cm}|p{3.8cm}|p{3.8cm}|}
    \hline
    \textbf{Nr} & \textbf{Disziplin} & \textbf{Risiko} & \textbf{Massnahmen PREN1} & \textbf{Massnahmen PREN2} \\ \hline
    1    &  Elektrotechnik    & Spannungsabfall/ Wackelkontakt bei Schleifkontakten & Stützkondensator/ Akku & weitere Schleifkontakte \\ \hline
    2    &  Elektrotechnik    & MC Überlastung durch Antriebsencoder                & aktuelle Messung zeigen 10\% Auslastung & Encoder tauschen \\ \hline
    3    & Elektrotechnik    & Strom-Pi mit Akku zu teuer                           & -  & alternative suchen \\ \hline
    4    & Elektrotechnik    & zu hoher Anlaufstrom beim Motor                      & -  & Momentenregelung mittels MC \\ \hline
    5    & Elektrotechnik    & Ultraschallsensor zu ungenau                         & - & Systematische Fehler mittels Software korrigieren / TOF Sensor verwenden \\ \hline
    6    & Maschinenbau      & zu wenig Reibung Räder => Schienen                   & Aufgummierung geplant & Oberfläche der Räder optimieren (Rauheit) \\ \hline
    7    & Maschinenbau     & Kupplungen schleifen                                  & Grössere Haftreibung durch erhöhte Rauheit & alternative Befestigungsmethode wählen (Gewindestifte) \\ \hline
    8    & Maschinenbau & Gewichtsverteilung Zug nicht optimal                      & statische Berechnungen durchgeführt & Gegengewichte einsetzten \\ \hline
    9   & Maschinenbau  & Kippen in der Kurve oder bei Würfelaufnahme               & Konzept mit Kraftausgleich ausgearbeitet & Alternatives Konzept verfolgen \\ \hline
    10  & Maschinenbau & Riemen defekt                                              & Ersatzriemen auf Lager            & Riemenart wechseln \\ \hline
    11  & Maschinenbau & Reibfläche zwischen Ladungsträger und Wägen                & Material mit günstigen Gleiteigenschaften verwendet & Axiallager einbauen \\ \hline
    12  & Maschinenbau & Würfel Aufnahme/ Platzierung                               & Optimierung des Konzeptes «Greifers» & alternatives Konzept verfolgen \\ \hline
    13  & Maschinenbau & Reibung zu hoch bei Kran                                   & Material mit günstigen Gleiteigenschaften verwendet & Lagerung einbauen \\ \hline
    14 & Informatik    & Tafelerkennung zu langsam                              & Optimierung der Algorithmen   & wechsel auf Raspberry PI 3 A+ \\ \hline
    15 & Informatik     & Nummererkennung zu langsam                            & Optimierung der Algorithmen   & wechsel der ML- Framework \\ \hline
    16 & Informatik     & Kommunikation MC <=> RPI zu langsam                   & max. Baudrate verwendet       & wechsel zu anderem Bus \\ \hline
    17 & Informatik     & Bildverarbeitung zu langsam                           & C++ als Programiersprache verwenden & OpenCL verwenden \\ \hline
    \caption{Risikotabelle}
    \end{longtable}


\subsection{Nächste Schritte}
Nach der Konzeptionsphase im PREN1 folgt nun die Realisation des Zugs im PREN2. Das Ziel am Anfang des PREN2 wird sein, so schnell wie möglich die mechanische Grundplattform des Zuges zu erstellen, damit die Elektrontik und die Software getestet werden können. Währendessen arbeiten Elektrotechnik und Informatik an der finalisierung ihrer Komponenten. Mitte des PREN2 ist geplant, den fertigen Zug in physischem Zustand fertig zu haben, damit in der zweiten Hälfte das Testen und Optimieren beginnen kann.
\paragraph{Elektrotechnik}
\begin{enumerate}
    \item ersten Prototyp erstellen (damit Informatiker testen können)
        \subitem Aufbau auf erstem Zug Prototyp vom Maschinenbau
        \subitem PCB für Stromversorgung und Antrieb
        \subitem Microkontroller mit Software für Antrieb und Kommunikaiton mit Pi
    \item Ansteuerung / Auslesen Sensoren
    \item Ansteuerung Schwenker
    \item Anpassungen (speziell für präzisen Halt)
    \item Kompletter Prototyp
        \subitem PCB für alle Komponenten
        \subitem Komplette Software für Mikrocontroller
    \item Finales Zusammenführen mit Informatik und Maschinenbau
\end{enumerate}

\paragraph{Informatik}
\begin{enumerate}
    \item Finalisieren bestehender Komponenten
        \subitem Signalerkennung, Gleiserkennung, Akustik, Sensorik
    \item Die restlichen Komponenten implementieren
        \subitem Interface zu Motor/ Kran
    \item Hauptablauf implementieren
    \item Testing \& Bugfixing
\end{enumerate}

\paragraph{Maschinenbau}
\begin{enumerate}
    \item Konstruktionszeichnungen erstellen
    \subitem Funktionsmasse
    \subitem Materialwahl
    \item Fertigen eines voll funktionsfähigen Protoyps, damit die Elektronik angebracht werden kann und die Informatik ihre Versuche durchführen kann.
    \subitem Einkaufsteile bestellen
    \subitem Fertigung durch Werkstattpersonal oder Eigenfertigung
    \item Weitere Versuche an Prototyp durchführen sowie Optimierungen und Änderungen vornehmen
\end{enumerate}

\subsection{Lessons learned}

\paragraph{Lessons learned Elektronik}
Es hat sich gezeigt, dass eine enge Zusammenarbeit unter den Disziplinen entscheidend ist. Ein sehr gutes Beispiel dafür ist die Auswahl des Motors. Weder die Fachleute Elektrotechnik, noch die Fachleute Maschinentechnik können diese Auswahl alleine treffen. Bevor eine genauere Auswahl vorgenommen werden kann müssen zuerst von beiden Disziplinen alle Anforderungen und Limitierungen zusammengetragen und diskutiert werden. Im Team kann diese Auswahl dann sehr gut getroffen werden.

\paragraph{Lessons learned Informatik}
Für die Informatik ist es schwierig etwas Greifbares zu erschaffen ohne die Mithilfe von den anderen Disziplinen. Deshalb
war die grösste Lektion, dass eine gute Softwarearchitektur Goldwert ist. In unserem Fall konnten wir dank der
Middleware Architektur schon finale Komponenten entwickeln und diese testen obwohl der Zug noch nicht vollständig gebaut
ist. Dies gilt speziell für die Signalerkennung, Gleiserkennung, Akustik und die Sensorik.

\paragraph{Lessons learned Maschinenbau}
Ein grosser Vorteil im Bereich der Maschinentechnik sind die Erfahrungen, welche die Teammitglieder der Mechanik durch ihre Ausbildung als Konstrukteure einbringen können. Dadurch sind rasch gute Lösungskonzepte entwickelt und ausgearbeitet worden. Teilkonzepte konnten somit bereits mit Prototypen getestet werden. Es wurde die Erfahrung gemacht, dass für die Maschinentechnik die Zusammenarbeit mit den anderen Disziplinen wesentlich ist, da sie das Grundgerüst des Zuges bildet und die Verbindung zwischen Elektronik und Informatik ist.
\end{document}