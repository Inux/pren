\documentclass[../../main.tex]{subfiles}
\begin{document}
\subsection{Schlussdiskussion}

- Entwicklungskosten, zeitlicher Entwicklungsaufwand
- Erfahrungen, „Lessons learned“, kritische Würdigung der Arbeiten
- offene Punkte, Risiken und Ausblick auf PREN 2


\subsection{Risiken ET}
\begin{table}[H]
    \begin{tabular}{|p{0.3\textwidth}|p{0.3\textwidth}|p{0.4\textwidth}|}
    \hline
    \textbf{Risiko}                                            & \textbf{Massnahmen ergriffen}                                                                         & \textbf{Mögliche Massnahme bei weiteren Problemen}                                                                \\ \hline
    Spannungsabfall / Wackelkontakt bei Schleifkontakten       & Stützkondensator eingeplant / Systemstuerung durch Akku gestützt                                      & weitere Schleifkontakte hinzufügen                                                                                \\ \hline
    Mikrocontroller Überlastet durch Antriebsencoder-Interrupt & Messung zeigt Maximalgeschwindigkeit wird zu ca. 10\% Auslastung mit Antriebsencoder-Interrupt führen & Encoder mit weniger Impulsen einsetzten / Frequenzteiler zwischen Encoder und Microkontroller einsetzen           \\ \hline
    Strom-Pi mit Akku ist zu teuer                             &                                                                                                       & Alternative Komponenten finden                                                                                    \\ \hline
    Motor benötigt zu hohen Anlaufstrom                        &                                                                                                       & sanftes Anfahren mit dem Microkontroller                                                                          \\ \hline
    HC-SR04 misst die Distanz nicht präzise genug              &                                                                                                       & Systematische Fehler in der Software korrigieren / Bei zu hohen zufälligen Fehler auch ein TOF-Sensor einsetzten  \\ \hline
    \end{tabular}
    \caption{Risiken Elektrotechnik}
    \label{tab:risk_et}
    \end{table}

    \subsection{Risiken Informatik}
    \begin{table}[H]
        \begin{tabular}{|p{0.3\textwidth}|p{0.3\textwidth}|p{0.4\textwidth}|}
    \hline
    \textbf{Risiko}          & \textbf{Massnahmen ergriffen}                                                                                       & \textbf{Mögliche Massnahme bei weiteren Problemen} \\ \hline
    \end{tabular}
    \caption{Risiken Informatik}
    \label{tab:risk_I}
    \end{table}

    \subsection{Risiken Maschinenbau}
    \begin{table}[H]
        \begin{tabular}{|p{0.3\textwidth}|p{0.3\textwidth}|p{0.4\textwidth}|}
        \hline
        \textbf{Risiko}   & \textbf{Massnahmen ergriffen}                                                                                   & \textbf{Mögliche Massnahme bei weiteren Problemen} \\ \hline
        Zu wenig Reibung zwischen Räder und Schiene & Aufgummierung geplant & Oberflächenbeschaffenheit des Rades selber optimieren (Rauheit erhöhen) \\ \hline
        Kupplungen schleifen & Grössere Haftreibung durch erhöhte Rauheit erzielen & Alternative Befestigungsmethode wählen (mit Gewindestift anschrauben) \\ \hline
        Gewichtsverteilung von Lokomotive nicht optimal & Statische Berechnung vorlaufend ergänzen, wenn Anpassungen an Prototyp vorgenommen werden & Gegengewichte einsetzen \\ \hline
        Kippen in der Kurve bzw. bei Würfelaufnahme & Zu beginn im Team abgesprochen Vernachlässigung des Konzeptes der Kraftausgleich bei Würfelaufnahme und Kurven fahrt wieder aufgreifen & Alternatives Konzept verfolgen \\ \hline
        Riemen defekt & Ersatzriemen auf Lager & Riemenart wechseln \\ \hline
        Rütteln bei Fahrt (Kamera) & Zu beginn im Team abgesprochen Vernachlässigung des Konzeptes der Fahrzeugdämpfung wieder aufgreifen & Nur Kameras dämpfen \\ \hline
        Schleifkontakte funktionieren nicht & Optimierung an Konstruktion vornehmen & Alternative Schleifkontakte verwenden \\ \hline
        Reibfläche zwischen Ladungsträger und Wägen (zu grosse Reibkräfte) & Material mit günstigen Gleiteigenschaften &  Axiallager einbauen\\ \hline
        Würfel kann nicht aufgenommen und / oder korrekt platziert werden & Optimierung des Konzepts "Greifer" &  Alternatives Konzept verfolgen\\ \hline
        Würfel ist zu wenig gesichert & Spannvorrichtung vorgesehen &  Alternative Befesitungsmethode anwenden\\ \hline
        Reibung ist zu hoch bei Kran & Material mit günstigen Gleiteigenschaften &  Lagerung einbauen\\ \hline
        \end{tabular}
        \caption{Risiken Maschinenbau}
        \label{tab:risk_M}
        \end{table}

\subsection{Nächste Schritte ET}
\begin{enumerate}
    \item ersten Prototyp erstellen (damit Informatiker testen können)
        \subitem Aufbau auf erstem Zug Prototyp vom Maschinenbau
        \subitem PCB für Stromversorgung und Antrieb
        \subitem Microkontroller mit Software für Antrieb und Kommunikaiton mit Pi
    \item Ansteuerung / Auslesen Sensoren
    \item Ansteuerung Schwenker
    \item Anpassungen (speziell für präzisen Halt)
    \item Kompletter Prototyp
        \subitem PCB für alle Komponente
        \subitem Komplette Software für Mikrocontroller
    \item Finales zusammenführen mit Informatik und Maschinenbau
\end{enumerate}

\subsection{Nächste Schritte Informatik}
\begin{enumerate}
    \item Finalisieren bestehender Komponenten
        \subitem Signalerkennung
        \subitem Gleiserkennung
        \subitem Akustik
        \subitem Sensorik
    \item Die restlichen Komponenten implementieren
        \subitem Interface zu Motor
        \subitem Interface zu Kran
    \item Hauptablauf implementieren
    \item Testing \& Bugfixing
\end{enumerate}

\subsection{Nächste Schritte Maschinenbau}
\begin{enumerate}
    \item Konstruktionszeichnungen erstellen
    \subitem Funktionsmasse
    \subitem Materialwahl
    \item Fertigen eines voll funktionsfähigen Protoyps, damit die Elektronik angebracht werden kann und die Informatik ihre Versuche durchführen kann.
    \subitem Einkaufsteile bestellen
    \subitem Fertigung durch Werkstattpersonal oder Eigenfertigung
    \item Weitere Versuche an Prototyp durchführen sowie Optimierungen und Änderungen vornehmen
\end{enumerate}

\subsection{Lessons learned ET}
Es hat sich gezeigt, dass eine enge Zusammenarbeit unter den Disziplinen entscheidend ist. Ein sehr guten Beispiel dafür ist die Auswahl des Motors. Weder die Fachleute Elektrotechnik, noch die Fachleute Maschinentechnik können diese Auswahl alleine treffen. Bevor eine genauere Auswahl vorgenommen werden kann müssen zuerst von beiden Disziplinen alle Anforderungen und Limitierungen zusammengetragen und diskutiert werden. Im Team kann diese Auswahl dann sehr gut getroffen werden.

\subsection{Lessons learned Informatik}
Für die Informatik ist es schwer etwas greifbares zu erschaffen ohne die Mithilfe von den anderen Disziplinen. Deshalb
war die grösste Lektion, dass eine gute Softwarearchitektur Goldwert ist. In unserem Fall konnten wir Dank der
Middleware Architektur schon finale Komponenten entwickeln und diese Testen obwohl der Zug noch nicht vollständig gebaut
ist. Dies gilt speziell für die Signalerkennung, Gleiserkennung, Akustik und die Sensorik

\subsection{Lessons learned Maschinenbau}
Ein grosser Vorteil im Bereich der Maschinentechnik sind die Erfahrungen, welche die Teammitglieder der Mechanik durch ihre Ausbildung als Konstrukteure einbringen können. Dadurch sind rasch gute Lösungskonzepte entwickelt und ausgearbeitet worden. Teilkonzepte konnten somit bereits mit Prototypen getestet werden. Es wurde die Erfahrung gemacht, dass für die Maschinentechnik die Zusammenarbeit mit den anderen Disziplinen wesentlich ist, da sie das Grundgerüst des Zuges bildet und die Verbindung zwischen Elektronik und Informatik ist.
\end{document}