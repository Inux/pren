\documentclass[../../main.tex]{subfiles}
\begin{document}

\begin{comment}
    Schlussdiskussion mit
    - Entwicklungskosten, zeitlicher Entwicklungsaufwand
    - Erfahrungen, „Lessons learned“, kritische Würdigung der Arbeiten
    - offene Punkte, Risiken und Ausblick 
\end{comment}

\subsection{Kosten}
Gemäss der Aufgabenstellung steht ein Budget von 500 Franken zur Verfügung. In Tabelle \ref{tab:kosten_total} wurden alle Komponenten in einer Kostenübersicht zusammengetragen.\\
In dieser Übersicht nicht enthalten sind Komponenten, welche aus der Mechanikwerkstatt oder dem Elektronik Labor der
HSLU bezogen werden können. Auch sind Stundenansätze für Werkstattpersonal oder 3D-Drucker nicht berücksichtigt. \\

\begin{table}[H]
    \centering
    \begin{tabular}{|p{6cm}|l|r|r|r|}
    \hline
    \textbf{Beschreibung}                                   & \textbf{Lieferant} & \textbf{Anzahl} & \textbf{Stückpreis} & \textbf{Gesammtpreis} \\ \hline 
    Raspberry Pi 3 Model B+                                 & pi-shop            & 1               & CHF 39.00           & CHF 39.00             \\ \hline 
    Raspberry Pi Zero W                                     & pi-shop            & 1               & CHF 10.80           & CHF 10.80             \\ \hline 
    Raspberry Pi Kamera                                     & pi-shop            & 2               & CHF 32.90           & CHF 65.80             \\ \hline 
    Passiver Buzzer                                         & play-zone          & 1               & CHF 5.00            & CHF 5.00              \\ \hline 
    StromPi 3                                               & Conrad             & 1               & CHF 49.95           & CHF 49.95             \\ \hline 
    StromPi 3 Battery Pack, 1000 mAh                        & Conrad             & 1               & CHF 31.95           & CHF 31.95             \\ \hline 
    Beschleunigungssensor ADXL345                           & aliexpress         & 1               & CHF 1.00            & CHF 1.00              \\ \hline 
    Tiny K22                                                & hslu               & 1               & CHF 26.00           & CHF 26.00             \\ \hline 
    Arduino IBT\_2 (DC-Motoren Treiber) (43A)               & wish               & 1               & CHF 9.00            & CHF 9.00              \\ \hline 
    DC-DC CC CV Buck Converter (12A)                        & wish               & 1               & CHF 3.00            & CHF 3.00              \\ \hline 
    Ultraschallsensor HS-SR04                               & aliexpress         & 1               & CHF 0.77            & CHF 0.77              \\ \hline 
    TOF Sensor VL53L0X                                      & aliexpress         & 1               & CHF 3.69            & CHF 3.69              \\ \hline 
    LM2596 DC-DC Schritt-down Converter (2.5A)              & aliexpress         & 1               & CHF 0.90            & CHF 0.90              \\ \hline 
    Logic-Level-Converter Bi-Directional Modul 5 v zu 3,3 v & aliexpress         & 1               & CHF 0.90            & CHF 0.90              \\ \hline 
    Tower Pro Micro Servo SG90                              & wish               & 1               & CHF 1.60            & CHF 1.60              \\ \hline 
    Mini Modul PWM Speed Control Über L298N                 & aliexpress         & 1               & CHF 0.69            & CHF 0.69              \\ \hline 
    Antriebsmotor mit Encoder                               & Sponsor            & 1               & CHF 30.00           & CHF 30.00             \\ \hline 
    Schwenker Motor mit Encoder                             & Sponsor            & 1               & CHF 25.00           & CHF 25.00             \\ \hline 
    Zahnriehmen                                             & Mädler             & 2               & CHF 13.19           & CHF 26.38             \\ \hline 
    Zahnriehmenrad                                          & Mädler             & 3               & CHF 8.00            & CHF 24.00             \\ \hline 
    Kupplungen                                              & Mädler             & 2               & CHF 30.00           & CHF 60.00             \\ \hline 
    Zahnräder                                               & Mädler             & 1               & CHF 14.35           & CHF 14.35             \\ \hline 
    Zahnräder                                               & Mädler             & 1               & CHF 3.25            & CHF 3.25              \\ \hline 
    Zahnräder                                               & Mädler             & 1               & CHF 4.77            & CHF 4.77              \\ \hline \hline
    \textbf{Total}                                          & \textbf{}          & \textbf{29}     & \textbf{CHF 345.71} & \textbf{CHF 437.80}   \\ \hline 
    \end{tabular}
    \caption{Kostenübersicht Gesamtkonzept}
    \label{tab:kosten_total}
    \end{table}

    \nocite{PiShopPi3ModelBp}
    \nocite{PiShopPiZero}
    \nocite{PiShopBrightPi}
    \nocite{PiShopPiKamera}
    \nocite{PlayZonePassiverBuzzer}
    \nocite{ReicheltStromPi}
    \nocite{ReicheltStromPiAkku}
    \nocite{AliExpressADXL345}
    \nocite{WisIBT2}
    \nocite{WishDCDCConverter}
    \nocite{AliExpressHCSR04}
    \nocite{AliExpressVL53L0X}
    \nocite{AliExpressLM2596}
    \nocite{Distrelec13FR200E}
    \nocite{AliExpress5V33VConverter}
    \nocite{WishTowerpro48}
    \nocite{WishTowerpro48}
    \nocite{MadlerZahnriemen}
    \nocite{MadlerZahnriemenrad}
    \nocite{MadlerKupplung}
    \nocite{MadlerZahnrad}


\pagebreak

\subsection{zeitlicher Entwicklungsaufwand}
Die im Stundenplan gegebenen zwei Halbtage wurden vom Team genutzt um die Entwicklung des Projekts voranzutreiben. Zusätzlich wurden von allen Teammitgliedern diverse Abende und Wochenende investiert um die Ziele zu erreichen. Eine genau Stundenzusammenstellung wurde über die zwei Semester nicht geführt. Daher wird hier darauf verzichtet eine genaue Anzahl Stunden für den Entwicklungsaufwand zu nennen.\\

\subsection{Lessons learned}
Im Rahmen beider PREN Module konnten im ganzen Team viele Wertvolle Erfahrungen gesammelt werden. Es konnte viel in der jeweils eigenen Disziplin gelernt werden, jedoch auch im interdisziplinären Kontext konnten alle Teammitglieder viel Lernen. Durch die enge Zusammenarbeit der verschiedenen Disziplin konnte das Verstäntniss über die Arbeit der jeweils anderen Disziplin gestärkt werden.\\

\subsubsection{Elektrotechnik}
In der Elektronik findet man sich oft als Schnittstelle zwischen der abstrakten Softwarewelt der Informatik und der physikalischen Mechanik. Dieses Projekt konnte gut aufzeigen, wie eine ständige Kommunikation mit allen Teammitgliedern unerlässlich ist.\\
Dies erfordert auch eine saubere Planung um einen möglichst fliesenden Projektablauf sicherzustellen. Zum Beispiel sollte die Motorenansteuerung bereit sein um die Funktionsweise der Mechanik zu Testen sobald diese Produziert ist, andererseits sollte z.B. die Kommunikation vom Mikrocontroller zum Pi bereit sein, sobald die Informatik erste Software-Komponenten beireit hat.\\
Die Unerlässlichkeil dieser interdisziplinären Planung und Kommunikation ist wohl die wichtigste und grösste Lektion aus den PREN Modulen.

\subsubsection{Informatik}
%TODO

\subsubsection{Maschinenbau}
Im Maschinenbau wurde klar, dass eine gute Planung im Voraus einem sehr viel Ärger ersparen kann. Grundsätzlich wurde das von unserem Team auch so umgesetzt. Umso mehr Zeit man in der Konstruktion investiert, umso besser lässt sich das Ganze schlussendlich zusammenbauen und funktioniert optimaler weise auch. Ohne einige Anpassungen kamen auch wir nicht durch. Doch im Grossen und im Ganzen sind wir zufrieden mit unserer Vorarbeit. Schlussendlich kann man sagen, dass die Mechanik die Anforderungen mehr als erfüllt.
Wie schon im Elektrotechnikteil erwähnt lohnt es sich früh mit den anderen Disziplinen abzusprechen und deren Wünsche probiert in der Konstruktion einzubauen.



\subsection{offene Punkte}
Das Konzept aus PREN1 konnte genug gut umgesetzt werden damit das Team zuversichtlich ist, dass der Zug die Aufgabenstellung erfüllen kann. Es gibt noch einige Punkte, die man als Erweiterung umsetzen könnte. Auf einige Punkte, die nicht umgesetzt werden konnten wird in Kapitel \ref{main_bewertung} eingegangen. Die Umsetzung dieser offenen Punkte wäre aber noch mit einem bedeutenden Zeitaufwand verbunden und dies ist leider im Rahmen diese Moduls nicht mehr möglich.

\subsection{Ausblick}
Wie gut sich das Konzept bewährt wird sich bei einem Wettbewerb mit allen anderen Teams zeigen. Dabei müssen alle Teams mit ihrem Zug die Aufgabenstellung bewältigen und erhalten Punkte für jede erfüllte Teilaufgabe. Zusätzlich wird die gemessen Zeit mit allen Teams verglichen und bewertet.



\end{document}