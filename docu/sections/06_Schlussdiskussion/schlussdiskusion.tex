\documentclass[../../main.tex]{subfiles}
\begin{document}

\begin{comment}
    Schlussdiskussion mit
    - Entwicklungskosten, zeitlicher Entwicklungsaufwand
    - Erfahrungen, „Lessons learned“, kritische Würdigung der Arbeiten
    - offene Punkte, Risiken und Ausblick 
\end{comment}

\subsection{Kosten}
Gemäss der Aufgabenstellung steht ein Budget von 500 Franken zur Verfügung. In Tabelle \ref{tab:kosten_total} wurden alle Komponenten in einer Kostenübersicht zusammengetragen.\\
In dieser Übersicht nicht enthalten sind Komponenten, welche aus der Mechanikwerkstatt oder dem Elektronik Labor der
HSLU bezogen werden können. Auch sind Stundenansätze für Werkstattpersonal oder 3D-Drucker nicht berücksichtigt. \\

% TODO: Tabelle mit Kostenübersicht

\subsection{zeitlicher Entwicklungsaufwand}
Die im Stundenplan gegebenen zwei Halbtage wurden vom Team genutzt um die Entwicklung des Projekts voranzutreiben. Zusätzlich wurden von allen Teammitgliedern diverse Abende und Wochenende investiert um die Ziele zu erreichen. Eine genau Stundenzusammenstellung wurde über die zwei Semester nicht geführt. Daher wird hier darauf verzichtet eine genaue Anzahl Stunden für den Entwicklungsaufwand zu nennen.\\

\subsection{Lessons learned}
Im Rahmen beider PREN Module konnten im ganzen Team viele Wertvolle Erfahrungen gesammelt werden. Es konnte viel in der jeweils eigenen Disziplin gelernt werden, jedoch auch im interdisziplinären Kontext konnten alle Teammitglieder viel Lernen. Durch die enge Zusammenarbeit der verschiedenen Disziplin konnte das Verstäntniss über die Arbeit der jeweils anderen Disziplin gestärkt werden.\\

\subsubsection{Elektrotechnik}
In der Elektronik findet man sich oft als Schnittstelle zwischen der abstrakten Softwarewelt der Informatik und der physikalischen Mechanik. Dieses Projekt konnte gut aufzeigen, wie eine ständige Kommunikation mit allen Teammitgliedern unerlässlich ist.\\
Dies erfordert auch eine saubere Planung um einen möglichst fliesenden Projektablauf sicherzustellen. Zum Beispiel sollte die Motorenansteuerung bereit sein um die Funktionsweise der Mechanik zu Testen sobald diese Produziert ist, andererseits sollte z.B. die Kommunikation vom Mikrocontroller zum Pi bereit sein, sobald die Informatik erste Software-Komponenten beireit hat.\\
Die Unerlässlichkeil dieser interdisziplinären Planung und Kommunikation ist wohl die wichtigste und grösste Lektion aus den PREN Modulen.

\subsubsection{Informatik}
%TODO

\subsubsection{Maschinenbau}
%TODO

\subsection{offene Punkte}
Das Konzept aus PREN1 konnte genug gut umgesetzt werden damit das Team zuversichtlich ist, dass der Zug die Aufgabenstellung erfüllen kann. Es gibt noch einige Punkte, die man als Erweiterung umsetzen könnte. Auf einige Punkte, die nicht umgesetzt werden konnten wird in Kapitel \ref{main_bewertung} eingegangen. Die Umsetzung dieser offenen Punkte wäre aber noch mit einem bedeutenden Zeitaufwand verbunden und dies ist leider im Rahmen diese Moduls nicht mehr möglich.

\subsection{Ausblick}
Wie gut sich das Konzept bewährt wird sich bei einem Wettbewerb mit allen anderen Teams zeigen. Dabei müssen alle Teams mit ihrem Zug die Aufgabenstellung bewältigen und erhalten Punkte für jede erfüllte Teilaufgabe. Zusätzlich wird die gemessen Zeit mit allen Teams verglichen und bewertet.



\end{document}