\documentclass[../../main.tex]{subfiles}
\begin{document}
\subsection{Management Summery}
Im Rahmen des Moduls PREN (Produktenwicklung) an der Hochschule Luzern wurde dieses Dokument durch das Team 28 erstellt. Das interdisziplinäre Team besteht aus Maschinenbau-, Elektrotechnik- und Informatikstudenten. In dieser Durchführung des Moduls wird in der Aufgabenstellung eine Entwicklung eines Schnellzuges gefordert. Dieser Zug muss eine definierte Strecke mit Geraden und Kurven in zwei Runden so schnell wie möglich passieren. Dabei muss der Zug einen Holzwürfel, welcher in der Startzone mittels einem Kran aufgeladen wird, während der gesamten Strecke transportieren. Während die Strecke zurückgelegt wird, muss der Zug eine Signaltafel mit Nummer erkennen und in der dritten Runde bei dieser Nummer so genau wie möglich anhalten. Die erkannte Nummer soll akkustisch ausgegeben werden. Der Zug muss autonom, also benutzerunabhängig, agieren. Im Rahmen des PREN1 wurden mit Technologierecherchen mehrere Lösungsvarianten erarbeitet. Aus diesen Lösungsvarianten wurde dann die optimale Kombination von Teilkomponenten ausgewählt und zu einem Gesamtlösungskonzept zusammengeführt.\\
Dieses Konzept beinhaltet eine mechanische Grundkonstruktion mit zwei Trägerwagen. Diese mechanische Grundkonstruktion wurde so entwickelt, dass der Schwerpunkt tief gehalten wird und somit hohe Kurvengeschwindigkeiten erreicht werden können. Angetrieben wird der Schnellzug mit einem DC- Motor mit eingebauter Encoderscheibe um präzise Geschwindigkeiten zu erreichen. Der Kran um den Würfel auf den Zug zu heben ist eine eigene Hub- Konstruktion mit DC- Motor. Die Signal- und Nummererkennung wird mittels Kamera und Machine- Learning Algorithmen realisiert. Als zentrale Recheneinheit kommt ein Raspberry PI 3+, welcher einem Raspberry PI Zero und einem Mikrocontroller unterstützt wird. Im PREN2 wird das entwickelte Konzept realisiert, getestet und optimiert.
\pagebreak
\end{document}