\documentclass[../../main.tex]{subfiles}
\begin{document}


\subsection{Management Summery}
Im Rahmen des Moduls PREN (Produktentwicklung) an der Hochschule Luzern wurde dieses Dokument durch das Team 28
erstellt. Dieses interdiszipliäre Team bestehend aus Maschinenbau-, Elektrotechnik-, und Informatikstudenten erhielt die Aufgabenstellung einen Schnellzug zu entwickeln. Dieser Zug muss eine definierte Strecke mit Geraden und Kurven in zwei Runden so schnell wie möglich passieren. Dabei muss der Zug eine Fracht, welche in der Startzone mittels eines Krans aufgeladen wird, während der gesamten Strecke transportieren. Während die Strecke zurückgelegt wird, muss der Zug eine Signaltafel mit Nummer erkennen und in der dritten Runde bei dieser Nummer so genau wie möglich anhalten. Das Konzept des Zuges wurd eim Vorgängermodul PREN1 entwickelt. Mit Hilfe von Technologierecherchen, Varianten und ersten Machbarkeitsanalysen konten mehrere Lösungsansätze in PREN 1 entwickelt werden. Durch das Zusammenführen der optimalen Lösungsvarianten konnte schlussendlich das finale Lösungskonzept erstellt werden. Im PREN 2 wurde nun dieses Lösungskonzpet als Prototyp \textit{umgesetzt}. Der Prototyp beinhaltet eine mechanische Grundkonstruktion eines Zuges, Elektronik und Sensorik für die Aktoren und Sensoren so wie Software für den Ablauf und das Zusammenspiel der einzelnen Komponenten. Nach der Umsetzung wurden umfassende \textit{Tests}, welche sorgefältig geplannt worden sind, druchgeführt. Diese stellten sicher, dass der Zug die geforderte Zuverlässigkeit erreicht. Vortlaufend wurden Erkentnisse aus den Tests als \textit{Optimierungsvorschläge} zurück in die Umsetztung gespiesen. Dies führte zu einer stetigen Verbesserung des Zuges, so dass am Schluss des PREN 2 einen funktionierenden Prototyp erstellt werden konnte.
\end{document}
